\documentclass[a4paper,12pt]{article}

%===================================================================================
% Paquetes
%-----------------------------------------------------------------------------------
\usepackage{amsmath}
\usepackage{float}
\usepackage{amsfonts}
\usepackage{amssymb}
\usepackage[utf8]{inputenc}
\usepackage{listings}
\usepackage[pdftex]{hyperref}
\usepackage{graphicx}

\usepackage{listings}
\usepackage{color}

\definecolor{dkgreen}{rgb}{0,0.6,0}
\definecolor{gray}{rgb}{0.5,0.5,0.5}
\definecolor{mauve}{rgb}{0.58,0,0.82}
\def\code#1{\texttt{#1}}

\lstset{frame=tb,
  language=Python,
  aboveskip=3mm,
  belowskip=3mm,
  showstringspaces=false,
  columns=flexible,
  basicstyle={\small\ttfamily},
  numbers=none,
  numberstyle=\tiny\color{gray},
  keywordstyle=\color{blue},
  commentstyle=\color{dkgreen},
  stringstyle=\color{mauve},
  breaklines=true,
  breakatwhitespace=true,
  tabsize=3
}

%-----------------------------------------------------------------------------------
% Configuración
%-----------------------------------------------------------------------------------
\hypersetup{colorlinks,%
	    citecolor=black,%
	    filecolor=black,%
	    linkcolor=black,%
	    urlcolor=blue}


\begin{document}

 

\title{ej6}

\begin{titlepage}
	\centering
	\vspace*{\fill}
	\vspace*{0.5cm}
	\huge\bfseries
	Modelo de Red Meta Poblacional\\
	\vspace*{2cm}
	Manual de Usuario\\
	\vspace*{\fill}
\end{titlepage}

\section*{Instrucciones de uso}
Al abrir la página se muestra primeramente una pequeña sección de ayuda (Fig.1) que describe los parámetros necesarios a rellenar para la simulación.

\begin{figure}[H]
	\centering
	\includegraphics[width=0.9\linewidth]{./1}
	\caption{sección de ayuda}
	\label{Fig.1}
\end{figure}

A continuación se muestran los campos a rellenar (Fig.2)

\begin{figure}[H]
	\centering
	\includegraphics[width=0.9\linewidth]{./2}
	\caption{Campos de entrada}
	\label{Fig.2}
\end{figure}

Para la configuración de la red es necesario escoger el tipo de meta modelo e importar (poniendo la dirección local del archivo) una red preestablecida que puede ser de extensión .json o .xml (Más adelante se explica el formato que deben cumplir los archivos en cada caso). Luego de establecer un nombre, puede compilar la red.\\

Para importar los parámetros de los nodos, puede generar una plantilla o importar directamente una previamente generada.\\
 Para generar una plantilla de configuración de parámetros, seleccione la opción $Generate$ y proporcione una dirección destino. Una vez hecho esto, puede editar la plantilla con cualquier editor de texto para cambiar los parámetros convenientemente.\\
 Para cargar una configuración de parámetros seleccione la opción $Load$ y proporcione la direción del archivo en cuestión.\\
Independientemente del método de selección de parámetros utilizado, luego de establecer un tiempo de simulación, puede presionar el botón "Run Simulation" para iniciar la misma.\\

El resultado se muestra a continuación (Fig.3).

\begin{figure}[H]
	\centering
	\includegraphics[width=0.9\linewidth]{./3}
	\caption{Gráficas}
	\label{Fig.3}
\end{figure}

Se muestra una representación gráfica de la red establecida y una gráfica que expone el comportamiento de la simulación.\\
Los nodos de la red pueden ser desplazads a conveniencia y al seleccionar uno de llos, se muestra su comportamiento (Fig.4).

\begin{figure}[H]
	\centering
	\includegraphics[width=0.9\linewidth]{./4}
	\caption{Nodo seleccionado}
	\label{Fig.4}
\end{figure}



\section*{Formato de redes a importar}
\subsection*{.json}
A continuación se muestra un ejemplo de red sencilla.
\begin{lstlisting}
{
	"graph": 
	{
		"directed" : false,
		"nodes" :
		 {
			"1" : 
				{
					"label" : "node1",
					"metadata" : {
					"cmodel" : "SIR"
				}
			},
			"2" : 
				{
					"label" : "node2",
					"metadata" : {
					"cmodel" : "SIR"
				}
			}
		},
		"edges": 
		[
			{
				"source" : "1",
				"target" : "2",
				"metadata" : {"weight" : 0.5}
			},
			{
				"source" : "2",
				"target" : "1",
				"metadata" : {"weight" : 0.5}
			}
		]
	}
}
\end{lstlisting}
Se siguen las especificaciones de formato de grafos que se muestran \href{http://graphml.graphdrawing.org/}{aquí}.\\
Debe tener un objeto que tenga un objeto $graph$. Este último debe tener dos objetos, $nodes$ y $edges$.\\
En $nodes$ debe haber una serie de objetos cuyos nombres serán los $id's$ de cada nodo. Estos objetos tendrán un objeto $label$ cuyo valor es un string que representa el nombre del nodo. Además también tendrá un objeto $metadata$ que posee la información adicional necesaria (en este caso sólo es necesario $cmodel$ cuyo valor debe ser un string con el tipo de modelo).\\
En $edges$ se debe proporcionar una lista con las aristas de la red. Cada objeto de la lista tendrá un objeto $source$ y un objeto $target$ que deben tener los $id's$ del nodo fuente y destino respectivamente. En $metadata$ se debe tener un objeto $weight$ con el peso de la arista en cuestión cuyo valor debe ser un float.

\subsection*{.xml}

\section*{Formato de parámetros de entrada}
\subsection*{.json}
La plantilla generada tendrá la siguiente forma (se muestra un ejemplo básico).

\begin{lstlisting}
[
	{
		"id": "1",
		"label": "node1",
		"model": "SIR",
		"y": { "S": 999, "I": 1, "R": 0 },
		"params": { "beta": 0.2, "gamma": 0.1, "N": 1000 }
	},
	{
		"id": "2",
		"label": "node2",
		"model": "SIR",
		"y": { "S": 999, "I": 1, "R": 0 },
		"params": { "beta": 0.2, "gamma": 0.1, "N": 1000 }
	}
]
\end{lstlisting}
Cada objeto de la lista (lista de nodos) tendrá, además de sus objetos básicos, dos objetos $y$ y $params$. Estos tendrán a su vez una serie de objetos que representan los parámetros modificables de cada nodo. Cambie los valores de estos a conveniencia

\subsection*{.xml}
La plantilla generada tendrá la siguiente forma (se muestra un ejemplo básico).
\begin{lstlisting}
<nodes>
	<node1 id="1" label="node1" model="SIR">
		<y S="999" I="1" R="0"></y>
		<params beta="0.2" gamma="0.1" N="1000"></params>
	</node1>
		<node2 id="2" label="node2" model="SIR">
		<y S="999" I="1" R="0"></y>
	<params beta="0.2" gamma="0.1" N="1000"></params>
	</node2>
</nodes>
\end{lstlisting}
Similar al formato del .json. Modifique los atributos de los objetos $y$ y $params$ a conveniencia
\end{document}